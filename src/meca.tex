\section{Modèle physique du chariot}
\label{meca}
Pour construire le modèle, nous avons fait des hypothèses simplificatrices. Nous avons n'avons pas eu le temps de toutes les valider, mais on peut les plasser dans deux catégories:
\begin{itemize}
	\item celles que nous avons validé par la théorie, en partant des paramètres (dimensions, masses, distances...) et en faisant un raisonnement s'appuyant sur des résultats connus (théorèmes, tables de valeurs...)
	\item et celles que nous avons validé par des expériences sur notre prototype mécanique (voir \ref{protoMeca}).
\end{itemize}
Mais la validation des hypothèses sur le chariot élévateur par des mesures sur notre prototype mécanique nécessite de supposer que notre prototype mécanique est lui-même un bon modèle qui se comporte comme un chariot élévateur. Cette hypothèse est appuyée par des arguments théoriques. Elle permet en fait de généraliser l'ensemble des productions de notre TX (modèles mécaniques et numériques, voir \ref{proto}) à tous les chariots élévateurs, et à tous les véhicules verticaux, fins et avec un centre de masse haut.
