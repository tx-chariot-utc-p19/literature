\section{Conclusion}
\subsection{Travail de recherche}
Cette TX a permis de trouver des modèles qui n'étaient pas forcément déjà connus. Le calcul d'un chemin pour l'optimisation du contrôle d'un robot est un problème ancien sur lequel il y a déjà beaucoup de travaux; nous avons donc choisi un sujet extrêmement spécifique et simple pour nous permettre de partir de (quasiment) zéro et mener nos recherches jusqu'à un début d'application.\\
La production de nouvelles connaissances n'est certainement pas grande, puisque nous avons choisi un problème un peu connu, et que nous ne sommes pas chercheurs. Cependant, nous apportons des éléments de réponse à un problème très précis; si quelqu'un d'autre cherche à le résoudre, il pourra très rapidement utiliser ce rapport, le modèle mécanique, ou le modèle numérique. Voici un exemple simple: le problème de recherche du meilleur chemin dans une grille pour un véhicule pouvant tomber facilement dans un virage est très certainement un problème qui a été résolu par d'autres personnes avant nous; cependant il est très difficile de trouver directement la réponse à ce problème très précis par des travaux de documentation. Nos productions ont donc pour objectif de répondre directement aux besoins de personnes souhaitant automatiser des chariots élévateurs dans un hangar en grille.\\
Cependant, nos travaux sont très loin d'une application industrielle réelle pour de nombreuses raisons.
\subsection{Ouverture sur des applications}
Nous avons produits des modèles physiques et numériques qui mettent un peu en application les résultats que nous avons trouvé au cours de nos recherches. Il s'agit d'une trace concrète de nos travaux expérimentaux au même titre que ce rapport, mais ces modèles ne sont pas utilisables dans de vraies applications industrielles pour plusieurs raisons.
\paragraph{Hypothèses sur les modèles} Une hypothèse très forte que nous avons fait est que notre modèle mécanique se comporte comme un vrai chariot (voir \ref{hypProtoCommeChariot}). La validité de cette hypothèse peut être prouvée par des mesures sur un vrai chariot qui pourront ensuite être utilisées dans des calculs ou comparées aux mêmes mesures sur le prototype. Une fois cette hypothèse validée, on peut envisager de réutiliser les résultats du modèle mécanique. Sinon, il faut le refaire.\\
L'invalidité du modèle mécanique n'affecte pas le modèle numérique. Bien que le modèle numérique en soit dépendant (puisqu'il utilise des fonctions de coût en temps), les calculs et la théorie du modèle numérique ne font aucune hypothèse, excepté le fait qu'un virage a un coût non nul (donc l'enchaînement de segments droits est priviligié). En supposant qu'on adopte un modèle mécanique où le virage a un coût nul, le modèle numérique n'est plus adapté, mais il suffit de changer de modèle de calcul en utilisant les autres algorithmes que nous avons mentionnés (Dijkstra, génétique, Ford-Bellman...).\\
Nos différentes productions ont donc des sensibilités différentes aux hypothèses mécaniques que nous avons faites.
\paragraph{Limitations du champ des applications}Nous avons donc une hypothèse qui permet (si elle est vraie) de généraliser le modèle mécanique à tous les véhicules verticaux, fins et à centre de masse élevé. Le modèle numérique, lui, fonctionne uniquement dans un hangar en grille, où les virages ont un coût non nul. Ces hypothèses rendent nos productions spécifiques à des applications très particulières.
\paragraph{Difficultés d'implémentation} Comme nous l'avons mentionné plus haut et dans l'introduction, notre travail n'est pas suffisamment abouti pour une application industrielle. Il manque un travail de relecture sur des éléments théoriques que nous n'avons pas mentionnés dans ce rapport; un travail d'adaptation des modèles mécaniques aux véhicules qui seront réellement pilotés; un travail d'adaptation du code aux plateformes sur lesquelles il sera exécuté (optimisation, réécriture dans des langages bas-niveau); la conception d'un programme de contrôle (l'asservissement de la vitesse de notre prototype n'étant pas forcément adapté à des applications demandant des objectifs de position précis); la conception de systèmes odométriques permettant d'exécuter correctement les instructions venant du modèle numérique.\\
De plus, il existe des éléments à rajouter que nous ne pouvons pas prévoir parce qu'ils sont intrinsèques aux applications (par exemple, si des humains travaillent à proximité de robots, implémenter des règles de sécurité sur la vitesse, la détection d'obstacle...).
