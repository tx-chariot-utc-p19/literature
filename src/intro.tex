\section*{Introduction}

\subsection{Problématique}
La problématique initiale était la suivante: comment optimiser le trajet d’un chariot élévateur dans un entrepôt sans qu’il ne se retourne ou ne renverse les marchandises qu’il transporte ? Nous avons choisi de dérouler cette problématique jusqu'à isoler plusieurs problèmes précis. Ces problèmes sont interdépendants, mais leur résolution demande des méthodologies différentes.

Nous avons choisi de résoudre un problème d'optimisation, consistant à minimiser le temps de parcours d'une trajectoire dont seuls certains points sont définis. Cela signifie qu'il faut déterminer un chemin à coût minimal: c'est un problème de cheminement. Des algorithmes connus et prouvés permettent de résoudre de trouver des chemins optimaux étant donné des coûts, comme nous le verrons en \ref{combinaison}. Cependant, il reste le problème de déterminer la fonction de coût en temps.

L'approche la plus simple est de faire l'hypothèse que la vitesse est constante, donc que le coût en temps est égal au coût en distance. Cependant, cette approche est très insatisfaisante car les chariots élévateurs ne sont pas très stables:
\begin{itemize}
	\item Ils peuvent tomber dans un virage, si leur vitesse est trop élevée
	\item Ils peuvent faire tomber leur colis au cours d'un freinage, si leur décélération est trop importante
\end{itemize}
Le second point fait qu'il faut beaucoup de temps aux chariots pour décélérer. Par conséquent, la longueur d'un segment de droite n'est pas directement proportionnelle à la vitesse max qui pourra être atteinte; on ne peut "profiter de la ligne droite" car on ne peut accélérer sensiblement qu'à partir d'une certaine distance de voie libre pour freiner.
Le premier point fait que la vitesse du chariot élévateur dans un virage est sensiblement diminuée, ce qui fait qu'à distance égale, le temps nécessaire pour parcourir un segment de droite et une succession de virage est sensiblement différent.

Pour ces raisons, il est absolument nécessaire de modéliser le comportement du chariot sur certaines trajectoires. Les objectifs dégagés sont donc:
\begin{itemize}
	\item Déterminer la vitesse max du chariot garantissant:
		\begin{itemize}
			\item Un non-renversement du chariot en virage
			\item Un non-glissement du colis transporté en ligne droite
		\end{itemize}
	\item Déterminer le coût en temps d'une trajectoire donnée
	\item Sélectionner la meilleure trajectoire (en temps) passant par des points imposés
\end{itemize}

C'est la résolution de ces problèmes qui sera abordée dans ce rapport.


\subsection{Remerciements}
Nous tenons à remercier M. Jean-Luc DULON (Maître de Conférences Roberval FRE UTC-CNRS 2012 jean-luc.dulong@utc.fr);\\
et Mme. Delphine VIANDIER (auteur d'ouvrages et professeur de physique-chimie MP, lycée Pierre d'Ailly Compiègne delphine.viandier@worldonline.fr),\\
pour leur suivi pédagogique tout au long de ce projet.
Nous remercions également le lycée Pierre d'Ailly qui a financé notre prototype mécanique.\\
\subsection{Auteurs}
Anthony TORDJMANN, étudiant en MP au lycée Pierre d'Ailly de Compiègne (anthony.tordjmann@gmail.com)\\
Romain MALIACH-AUGUSTE, étudiant en TC04 à l'Université de Technologie de Compiègne (romain.maliach-auguste@etu.utc.fr)
